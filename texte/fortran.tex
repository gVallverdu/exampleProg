\chapter{Programmes en Fortran}

\section{Lire -- écrire -- compter}

\codeFortran{aire_rectangle.f90}{Calcul de l'aire d'un rectangle.}{aire_rec}

\codeFortran{reste.f90}{Calcul du reste de la division de deux entiers.}{reste}

\codeFortran{perimetre.f90}{Calcul du pérmiètre d'un cercle.}{perimetre}

\section{Faire une condition}

\codeFortran{le_plus_grand.f90}{Affiche le nombre le plus grand.}{le_plus_grand}

\codeFortran{discriminent.f90}{Calcul des racines d'un polynôme de degré 2.}{discriminent}

\codeFortran{racine_x.f90}{Calcul de la racine carré d'un nombre.}{racine}

\section{Répéter une action}

\codeFortran{calc_val_fonction_1.f90}{Illustration de l'utilisation d'une boucle \texttt{POUR}}{valfonction1}

\codeFortran{calc_val_fonction_2.f90}{Illustration de l'utilisation d'une boucle \texttt{TANT\_QUE}.}{valfonction2}

\section{Variable indicée}

\subsection{Déclaration statique des tableaux}

\codeFortran{liste_impair_static.f90}{Création d'une liste contenant les premiers entiers impairs (déclaration statique).}{impairs_stat}

\codeFortran{liste_random_static.f90}{Création d'une liste de nombres pseudo-aléatoires (déclaration statique).}{random_stat}

\subsection{Déclaration dynamique des tableaux}

\codeFortran{liste_impair_dyn.f90}{Création d'une liste contenant les premiers entiers impairs (déclaration dynamique).}{impairs_dyn}

\codeFortran{liste_random_dyn.f90}{Création d'une liste de nombres pseudo-aléatoires (déclaration dynamique).}{random_dyn}

\section{Sous programme}

\codeFortran{calc_val_fonction_3.f90}{Utilisation de l'écriture d'une fonction et de l'appel à cette fonction.}{valfonction3}

\section{Premiers petits programmes}

\codeFortran{factorielle.f90}{Calcul de la factorielle d'un nombre entier.}{factorielle}

\codeFortran{produit.f90}{Calcul d'un produit.}{produit}

\codeFortran{somme.f90}{Calcul d'une somme.}{somme}

\codeFortran{population.f90}{Calcul itératif de la décroissance d'une population.}{population}

\codeFortran{minmax.f90}{Recherche du maximum et du minimum dans une liste.}{minmax}

\codeFortran{marche_aleatoire.f90}{Marche aléatoire d'un point dans un plan 2D.}{marchealeatoire}

\codeFortran{pi.f90}{Calcul du nombre $\pi$ par la méthode Monte Carlo.}{pi}

\codeFortran{trapeze.f90}{Intégration par la méthode des trapèzes.}{trapeze}

\codeFortran{simpson.f90}{Intégration par la méthode de Simpson.}{Simpson}

\codeFortran{schmidt.f90}{Procédé d'orthogonalisation de Gramm-Schmidt.}{schmidt}    

