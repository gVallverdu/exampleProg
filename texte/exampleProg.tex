\documentclass{article}

\usepackage[utf8]{inputenc}
\usepackage[T1]{fontenc}
\usepackage[francais]{babel}
\AtBeginDocument{\renewcommand\labelitemi{$\bullet$}}
\AtBeginDocument{\renewcommand\labelitemii{$\circ$}}

\usepackage[top = 2.5cm, left = 2cm, right = 2cm, bottom = 2cm]{geometry}

% ------------------------------------------------
% minted package
% ------------------------------------------------

\usepackage{minted}

%\usemintedstyle{friendly}
\usemintedstyle{default}

%Styles:
%~~~~~~~
%* monokai:
    %This style mimics the Monokai color scheme.
%* manni:
    %A colorful style, inspired by the terminal highlighting style.
%* rrt:
    %Minimalistic "rrt" theme, based on Zap and Emacs defaults.
%* perldoc:
    %Style similar to the style used in the perldoc code blocks.
%* borland:
    %Style similar to the style used in the borland IDEs.
%* colorful:
    %A colorful style, inspired by CodeRay.
%* default:
    %The default style (inspired by Emacs 22).
%* murphy:
    %Murphy's style from CodeRay.
%* vs:
    
%* trac:
    %Port of the default trac highlighter design.
%* tango:
    %The Crunchy default Style inspired from the color palette from the Tango Icon Theme Guidelines.
%* fruity:
    %Pygments version of the "native" vim theme.
%* autumn:
    %A colorful style, inspired by the terminal highlighting style.
%* bw:
    
%* emacs:
    %The default style (inspired by Emacs 22).
%* vim:
    %Styles somewhat like vim 7.0
%* pastie:
    %Style similar to the pastie default style.
%* friendly:
    %A modern style based on the VIM pyte theme.
%* native:
    %Pygments version of the "native" vim theme.

\newcommand{\codeAlgobox}[3]{%
    \renewcommand\listingscaption{Algorithme}
    \begin{listing}[H]
        \inputminted[linenos, 
                     stepnumber = 1,
                     bgcolor = black!5,
                     fontsize = \footnotesize,
                     samepage]{algobox}{#1}%
        \caption{#2}
        \label{lst:#3}
    \end{listing}
}%


% ------------------------------------------------
% ------------------------------------------------
\begin{document}

\section{Lire / écrire / calculer}

\codeAlgobox{algo_txt/aire_rectangle.algo}{Calcul de l'aire d'un rectangle.}{aire_rec}
\codeAlgobox{algo_txt/reste.algo}{Calcul du reste de la division de deux entiers.}{reste}
\codeAlgobox{algo_txt/perimetre.algo}{Calcul du pérmiètre d'un cercle connaissant son
rayon.}{perimetre}

\section{Faire une condition}

\codeAlgobox{algo_txt/le_plus_grand.algo}{Affiche le nombre le plus grand.}{le_plus_grand}
\codeAlgobox{algo_txt/discriminent.algo}{Calcul des racines d'un polynôme de degré
2.}{discriminent}
\codeAlgobox{algo_txt/racine_x.algo}{Calcul de la racine carré d'un nombre s'il est
positif.}{racine}

\section{Répéter une action}

\codeAlgobox{algo_txt/calc_val_fonction_1.algo}{Utilisation d'une boucle POUR pour
calculer les valeurs d'une fonction.}{valfonction1}
\codeAlgobox{algo_txt/calc_val_fonction_2.algo}{Utilisation d'une boucle TANT\_QUE pour
calculer les valeurs d'une fonction.}{valfonction2}

\section{Variable indicée}

\codeAlgobox{algo_txt/liste_impair.algo}{Création d'une liste contenant les premiers
entiers impairs.}{impairs}
\codeAlgobox{algo_txt/liste_random.algo}{Création d'une liste de nombres
pseudo-aléatoires.}{random}

\section{Sous programme}

\codeAlgobox{algo_txt/calc_val_fonction_3.algo}{Utilisation d'une boucle TANT\_QUE pour
calculer les valeurs d'une fonction en appelant une fonction.}{valfonction3}

\section{Premiers petits programmes}

\codeAlgobox{algo_txt/factorielle.algo}{Calcul de la factorielle d'un nombre
entier.}{factorielle}
\codeAlgobox{algo_txt/produit.algo}{Calcul d'un produit.}{produit}
\codeAlgobox{algo_txt/somme.algo}{Calcul d'une somme.}{somme}
\codeAlgobox{algo_txt/population.algo}{Calcul itératif de la décroissance d'une
population.}{population}
\codeAlgobox{algo_txt/minmax.algo}{Recherche du maximum et du minimum.}{minmax}
\codeAlgobox{algo_txt/marche_aleatoire.algo}{Marche aléatoire d'un point dans un plan 2D.}{marchealeatoire}
\codeAlgobox{algo_txt/trapeze.algo}{Intégration par la méthode des trapèzes.}{trapeze}
\codeAlgobox{algo_txt/simpson.algo}{Intégration par la méthode de Simpson.}{Simpson}
    
\listoflistings

\end{document}
