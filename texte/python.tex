\chapter{Programmes en Python}

\codePython{structure_generale.py}{Structure générale adoptée pour l'écriture des programmes en Python}{struct_gene}

\section{Lire -- écrire -- compter}

\subsection{Versions sans fonction}

\codePython{simple/aire_rectangle.py}{Calcul de l'aire d'un rectangle.}{aire_rec}

\codePython{simple/reste.py}{Calcul du reste de la division de deux entiers.}{reste}

\codePython{simple/perimetre.py}{Calcul du pérmiètre d'un cercle.}{perimetre}

\subsection{Versions avec une fonction}

\codePython{aire_rectangle.py}{Calcul de l'aire d'un rectangle.}{aire_rec_f}

\codePython{reste.py}{Calcul du reste de la division de deux entiers.}{reste_f}

\codePython{perimetre.py}{Calcul du pérmiètre d'un cercle.}{perimetre_f}

\section{Faire une condition}

\subsection{Versions sans fonction}

\codePython{simple/le_plus_grand.py}{Affiche le nombre le plus grand.}{le_plus_grand}

\codePython{simple/discriminent.py}{Calcul des racines d'un polynôme de degré 2.}{discriminent}

\codePython{simple/racine_x.py}{Calcul de la racine carré d'un nombre.}{racine}

\subsection{Versions avec une fonction}

\codePython{le_plus_grand.py}{Affiche le nombre le plus grand.}{le_plus_grand_f}

\codePython{discriminent.py}{Calcul des racines d'un polynôme de degré 2.}{discriminent_f}

\codePython{racine_x.py}{Calcul de la racine carré d'un nombre.}{racine_f}

\section{Répéter une action}

\codePython{calc_val_fonction_1.py}{Illustration de l'utilisation d'une boucle \texttt{POUR}}{valfonction1}

\codePython{calc_val_fonction_2.py}{Illustration de l'utilisation d'une boucle \texttt{TANT\_QUE}.}{valfonction2}

\section{Variable indicée}

\codePython{liste_impair.py}{Création d'une liste contenant les premiers entiers impairs.}{impairs}

\codePython{liste_random.py}{Création d'une liste de nombres pseudo-aléatoires.}{random}

\section{Sous programme}

\codePython{calc_val_fonction_3.py}{Utilisation de l'écriture d'une fonction et de l'appel à cette fonction.}{valfonction3}

\section{Premiers petits programmes}

\codePython{factorielle.py}{Calcul de la factorielle d'un nombre entier.}{factorielle}

\codePython{produit.py}{Calcul d'un produit.}{produit}

\codePython{somme.py}{Calcul d'une somme.}{somme}

\codePython{population.py}{Calcul itératif de la décroissance d'une population.}{population}

\codePython{minmax.py}{Recherche du maximum et du minimum dans une liste.}{minmax}

\codePython{marche_aleatoire.py}{Marche aléatoire d'un point dans un plan 2D.}{marchealeatoire}

\codePython{pi.py}{Calcul du nombre $\pi$ par la méthode Monte Carlo.}{pi}

\codePython{trapeze.py}{Intégration par la méthode des trapèzes.}{trapeze}

\codePython{simpson.py}{Intégration par la méthode de Simpson.}{Simpson}

\codePython{schmidt.py}{Procédé d'orthogonalisation de Gramm-Schmidt.}{schmidt}    
