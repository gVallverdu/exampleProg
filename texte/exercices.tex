\chapter{Énoncé des exercices}

Ce chapitre introduit les exemples qui seront présentés par la suite sous forme d'algorithmes et dans les différents langages de programmation. Les énoncés ci-dessous peuvent être vue comme les énoncés d'exercices dont l'objectif serait d'écrire les programmes donnés dans les chapitres suivants.

\section{Lire -- écrire -- calculer}

\begin{enumerate}
    \item Écrire un programme qui demande à l'utilisateur la longueur et la largeur d'un rectangle et calcule l'aire.
    \item Écrire un programme qui calcule le reste de la division de deux entiers demandés à l'utilisateur.
    \item Écrire un programme qui calcule le périmètre d'un cercle après avoir demandé son rayon.
\end{enumerate}

\section{Faire une condition}

\begin{enumerate}
    \item Écrire un programme qui demande deux chiffres à l'utilisateur et affiche le plus grand.
    
    \item Écrire un programme qui calcule la racine carré d'un nombre demandé à l'utilisateur.
    
    \item Écrire un programme qui calcule les racines d'un polynôme du second degré.
    %
    \begin{align*}
        & a x^2 + b x + c = 0 \\
        & \Delta = b^2 - 4 a c \\
        &
        \begin{cases}
            \Delta > 0 \qquad x = \dfrac{-b \pm \sqrt{\Delta}}{2a} & \text{2 solutions réelles} \\
            \Delta = 0 \qquad x = \dfrac{-b}{2a} & \text{une unique solution} \\
            \Delta < 0 \qquad x = \dfrac{-b \pm i\sqrt{|\Delta|}}{2a} & \text{2 solutions complexes}
        \end{cases}
    \end{align*}
\end{enumerate}

\section{Répéter une action}

Les deux exercices suivants consistent à calculer les valeurs d'une fonction de votre choix dans l'intervalle $[-5;5]$ avec

\begin{enumerate}
    \item un pas de $1$.
    \item avec un pas de $0.5$.
\end{enumerate}

\section{Variable indicée}

Les deux exercices suivants ont pour objectif d'introduire l'utilisation des variables indicées ou tableaux.

\begin{enumerate}
    \item Écrire un programme qui crée une liste contenant les premiers entiers impairs et affiche cette liste.
    \item Écrire un programme qui crée une liste contenant une série de nombres pseudo-aléatoires et affiche cette liste.
\end{enumerate}

\section{Sous programmes}

Reprendre l'exercice consistant à calculer les valeurs d'une fonction dans l'intervalle $[-5;5]$ en créant une fonction qui sera appelée pour calculer les valeurs souhaitées.

Les autres exercices pourront être retravailler avec pour objectif de rendre fonctionnel les actions qu'ils réalisent.

\section{Premiers petits programmes}

Les exemples suivants constituent des cas plus concrets pouvant nécessiter d'associer l'écriture d'une condition et d'une boucle. Ils sont classés par ordre de complexité croissante.

\begin{enumerate}
    \item Écrire un programme qui calcule la factorielle d'un entier naturel.
    
    Rappel :
    \begin{equation*}
        \begin{cases}
            n! = 1 \times 2 \times 3 \times \ldots \times n & \forall n \in \mathbb{N}^* \\
            n! = 1 & n = 0
        \end{cases}
    \end{equation*}
    
    \item Écrire un programme qui calcule le produit des $n$ premiers entiers impairs.
    
    \item Écrire un programme qui calcule la somme des $n$ premiers entiers naturels.
    
    \item Une population décroit de 40\% tous les 3 ans. La population étant considérée négligeable lorsqu'elle est inférieure à 0.1\% de sa valeur initiale, au bout de combien d'année l'extinction est-elle atteinte ?
    
    \item Écrire un programme qui construit une liste de nombres compris entre 0 et 100 puis cherche le minimum et le maximum dans cette liste.
    
    \item La marche aléatoire d'un point peut être modélisée de la façon suivante : 
    %
    \begin{equation*}
        \vec{r}(t + dt) = \vec{r}(t) + a \hat{\vec{R}}
    \end{equation*}
    %
    où $a$ désigne l'amplitude du déplacement aléatoire et $\hat{\vec{R}}$ est un vecteur aléatoire dont les composantes sont comprises entre -1 et 1.
    
    Écrire un programme qui met en œuvre une marche aléatoire dans un espace à deux dimensions et affiche à chaque pas de temps les coordonnées du point.

    \item Le nombre $\pi$ peut être calculé par un processus dit de Monte Carlo, mettant en oeuvre le tirage de nombres aléatoires. Le principe est le suivant : La probabilité qu'un point, de coordonnées $(x, y)$, $x\in[0,1]$ et $y\in[0,1]$, choisi aléatoirement, soit dans un cercle de rayon 1 est égale au rapport de l'aire du quart de cercle de rayon 1 compris dans le carré de largeur 1 et l'aire de ce carré. Écrire un programme qui calcule le nombre $\pi$ par cette méthode.
    
    \item Écrire un programme qui calcule l'intégrale d'une fonction par la méthode des trapèze.
    
    \item Écrire un programme qui calcule l'intégrale d'une fonction par la méthode de Simpson.
    
    \item Écrire un programme qui met en œuvre le procédé d'orthogonalisation de Gramm-Schmidt dont voici une brève description : Soit $\vec{u}$ et $\vec{v}$ deux vecteurs quelconques, le vecteur $\vec{v}'$ orthogonal au vecteur $\vec{u}$ est obtenu par :
    %
    \begin{equation*}
        \vec{v}' = \vec{v} - \frac{(\vec{u}.\vec{v})}{||\vec{u}||} \vec{u}
    \end{equation*}
\end{enumerate}

    
