\addcontentsline{toc}{chapter}{Introduction}
\chapter*{Introduction}

La programmation consiste à écrire dans un langage strict et normalisé une succession
d'opérations qui devront être exécutées par un ordinateur pour réaliser une action. Il
faut alors distinguer deux choses : le découpage de l'action que l'on souhaite que
l'ordinateur réalise en une succession d'opérations élémentaires et l'écriture de ces
opérations élémentaires dans le langage de programmation.

La deuxième étape correspond au choix de la syntaxe et peut dépendre des besoins en terme
d'efficacité du programme ou des bibliothèques disponibles par exemple, ou plus
légèrement des goûts du programmeurs.

Dans l'apprentissage de la programmation, la première étape est cruciale et parfois la plus 
difficile à franchir. Elle correspond à ce qu'on appelle l'algorithmique. Lorsqu'on sait 
décomposer en opérations élémentaires les actions a réaliser, il suffit alors de les écrire 
en utilisant la syntaxe souhaitée ou parfois imposée.

C'est dans cette optique que s'inscrit ce projet qui s'appuie sur 
\textsc{AlgoBox}\footnote{\url{http://www.xm1math.net/algobox}}, 
un outil développé à l'intention des étudiants 
du secondaire avec pour objectif l'apprentissage de l'algorithmique. Il permet d'écrire
des petits programmes en faisant abstraction de la partie syntaxique et en se concentrant
donc sur l'algorithme.

Ce projet ne correspond donc pas à l'écriture de code source original. Il s'agit ici, de 
mettre à disposition un ensemble d'exemples fonctionnels de programmes écrit en premier
lieu en utilisant \textsc{AlgoBox}, puis, traduit dans 
différents langages de programmation (fortran, python et C).

Quelques exemples, de programmes plus aboutis sont également fournis pour illustrer des
notions telles que les sous-programmes ou la modularité non abordées dans \textsc{AlgoBox}.

\section*{Bibliographie}

\begin{itemize}
    \item Le site d'AlgoBox \url{http://www.xm1math.net/algobox/>}
    \item Le site d'OpenClassRooms \url{http://fr.openclassrooms.com/}.
    \item Le site developpez.com \url{http://www.developpez.com/}.
    \item Claude Delannoy, \textit{Programmer en Fortran 90: guide complet} (Eyrolles, Paris, 1997).
    \item Claude Delannoy, \textit{Programmer en langage C: cours et exercices corrigés} (Eyrolles, Paris, 2009).
    \item Vincent Le Goff, \textit{Apprenez À Programmer En Python, Livre du Zéro} (Simple IT, 2012).
\end{itemize}

